% !TEX encoding = UTF-8 Unicode
% !TEX TS-program = pdflatex
% !TEX spellcheck = en-GB
% !BIB TS-program = bibtex
% !TEX root = ../MAIN.tex

\section{Maths}
\subsection{Numbering equations}
\[
\begin{sistema}
	x & = 1 + 2 \\
	y & = f(x) - 3
\end{sistema}
\]

Sistema con UN SOLO numero + graffa: \cref{eq.unica}
\begin{equation}\label{eq.unica}
\begin{sistema}
	x & = 1 + 2  \\
	y & = f(x) - 3
\end{sistema}
\end{equation}

Sistema con un numero DIVERSO per ogni eq numerate separatamente (ma graffa stretta): \cref{eq.Sist} e \cref{eq.Sista}
\begin{sistemaSub}
		x & = 1 + 2 \label{eq.Sist} \\
		y & = f(x) - 3 \label{eq.Sista}
\end{sistemaSub}

Numero COMUNE per ogni equazione + sottonumerazione (una label per l'intero sistema + una label per ogni eq.): la \cref{eq.energyBalance} contiene sia \cref{eq.EB1} che \cref{eq.EB2}
\begin{subequations}\label{eq.energyBalance}
	\begin{align}
		X 	& = Y + 2 \label{eq.EB1} \\ 
		l & = Z^2 - 3 \label{eq.EB2}
	\end{align}
\end{subequations}

{\color{red} DA PERFEZIONARE ANCORA: (vorrei fare la stessa cosa che in quello prima, ma con un ambiente -- ora non riesco a mettere la label unica esterna e nemmeno a mettere la graffa bella)
\begin{sistemaX}
			X 	& = Y + 2 \label{eq.EB3} \\ 
	l & = Z^2 - 3 \label{eq.EB4}
\end{sistemaX}
}

\subsection{Vectors and matrices}
\[
\vet{v} = \vet{\omega} \cross \vet{r}
\]
\[
\vet{v} = \vet{\omega} \scal \vet{r}
\]
\[
\norm{x} \qquad \norm*{\int x^2}
\]
\[
\abs{X} \qquad \abs*{\int x^2}
\]
\[
\mat{\Omega} = 
\begin{matrice}
	\vertbar & x & \vertbar \\
	y		& z & -x  \\
	\vertbar & x & \vertbar \\
\end{matrice}
\qquad
\mat{A} =
\begin{matrice}
    \horzbar & a^{T}_{1} & \horzbar \\[2ex]		% increase spacing
	\horzbar & a^{T}_{2} & \horzbar \\
			& \vdots    &          \\
	\horzbar & a^{T}_{n} & \horzbar
\end{matrice}
\]

\subsection{Derivatives}
\[
\du y
\]
Using the same command (default: 1st order)
\[
\der{f}{x} \qquad \der[2]{f}{x}
\]
Se lo si vuole fare inline c'è sia del 1st order: $\derline{y}{x}$ che di ordine superiore: $\derline[3]{y}{x}$
\[
y
\]
Idem, ma per quelle parziali:
\[
\parz{f}{x} \qquad \parz[2]{f}{x}
\]
e inline: $\parzline{y}{x}$ oppure $\parzline[4]{y}{x}$
\[
\grad f \qquad 
\dive f \qquad 
\curl \vet{v} \qquad
\lap v
\]
Per chiarezza si possono anche mettere le \{ \} attorno alla funzione (non cambia nulla per il comando)
\[
\grad{f} \qquad 
\dive{f} \qquad 
\curl{\vet{v}} \qquad
\lap{v}
\]
Derivata materiale:
\[
\Der{e}{t}
\]

\subsection{Miscellaneous}
Ora i comandi \verb|\epsilon| e \verb|\phi| producono quelle belle
\[
\epsilon \qquad \phi
\]
al posto degli aborti (che, non so perché, sono il default)
\[
\varepsilon \qquad \varphi
\]

Versione calligrafica delle lettere, maiuscole e minuscole (di default non ci sono le minuscole):
\[
\mathcal{G} \qquad \mathcal{g}
\]
\[
2 + 5 \ji \qquad \tilde{x} \qquad x = \cost \qquad \hat{x}
\]
Uguale per definizione:
\[
Y \defi \Delta H / \Delta L
\]
Media
\[
\avg{z}
\]
Gruppo adimensionale
\[
\Rey
\]
Underbrace con testo grande (leggibile)
\[
\sotto{\der{x}{y} - 3 + 12x - 7}{y^2}
\]
Big bracket for evaluation of derivatives:
\[
\eval{\der{y}{x}}_{2x}
\]