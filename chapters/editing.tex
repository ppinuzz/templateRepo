% !TEX encoding = UTF-8 
% !TEX TS-program = pdflatex
% !TEX spellcheck = en-GB
% !BIB TS-program = bibtex
% !TEX root = ../MAIN.tex

\section{Editing}
Commentando una line in \verb|editingcommands.sty| si possono cancellare le note ROSSE a piè di pagina fatte con \verb|\nota{}|%
\nota{Utile per snellire i documenti}
mentre quelle NERE con \verb|\footnote{}| restano%
\footnote{Come sempre}

\colbox{Testo messo in evidenza, con le sue note normali%
	\footnote{Tipo questa}
	e quelle "a scomparsa"%
	\nota{Come questa}
}

{\color{red} Dovrei mettere una flag nel preambolo del MAIN e far leggere al comando \verb|\nota{}| in editingcommand.sty quella flag, così non devo aprire il file \verb|.sty| ogni volta}


Posso mettere una \lato{parola} in grassetto nel testo, scriverla in grassetto nel margine e mandarla all'indice con un solo comando

\nb posso fare gli NB

Or highlight easily \imp{important text} (has problems with line breaks tho)
