% !TEX encoding = UTF-8 Unicode
% !TEX TS-program = pdflatex
% !TEX spellcheck = en-GB
% !BIB TS-program = bibtex

\documentclass[a4paper]{article}
\usepackage[T1]{fontenc}
\usepackage[utf8]{inputenc}
\usepackage[english]{babel}
\usepackage{microtype}		% nice-looking fonts

% MISCELLANEOUS PACKAGES
\usepackage{comment}
\usepackage{enumerate}		% for roman numerals (i), (ii), (iii), ecc.
% IF QUOTES ARE NOT SET AS ``...'' IN TEXSTUDIO (or OVERLEAF is used):
% LaTeX reproduces "..." verbatim (like ''...'') instead of using the mirrored version, to avoid typing them explicitely each time I can use the csquotes package
\usepackage[italian=quotes]{csquotes}
\MakeOuterQuote{"}
\usepackage{mVersion}		% version number N.M
\setVersion{0}				% "principal number" N (for big changes)
%\increaseBuild				% increases M at each compilation (decomment at the end of the change to be recorded)
\usepackage{xparse}			% for parsing when creating user-defined commands with default options
%\setlength{\parindent}{0pt} % uncomment to remove automatic indentation of first line of paragraph


% MATHS
\usepackage{amsmath,amssymb,mathtools}
\usepackage{amsthm}							% for the "Good practice" text box
\usepackage{bm}								% bold font 
\usepackage{braket}							% nice-looking brakets for sets
\usepackage[overload]{empheq} 				% For braced-style systems of equations.


% TABLES
\usepackage{booktabs}					% rules (i.e. lines) to divide the rows of table
\usepackage{caption}					% captions
\usepackage{tabularx}					% user-defined columns
\usepackage{float}						% float option [H]
\usepackage{tikz}
\graphicspath{{./figures}} 				% Directory of the images
%\usepackage{rotating}					% for rotated tables
\usepackage{tablefootnote}              % footnotes in tables
\newcolumntype{L}{>{\ttfamily}{l}<{}}	% column with typewriter font applied 
\newcolumntype{Y}{>{$}{l}<{$}}			% left-aligned column in math mode (https://tex.stackexchange.com/questions/112576/)


% CODES
%\usepackage[dvipsnames]{xcolor}	% colours for the environments
%\usepackage{pythonhighlight}
%\usepackage{matlabstyle}				% {matlab} environment


% UNITS OF MEASURE (needed in NOMENCLATURE too)
\usepackage{siunitx}						% units, scientific notation \num{3.5e-4} and decimal separator
\sisetup{per-mode=symbol,					% uses the / instead of the ^(-1) (e.g. m/s instead of ms^-1)
	range-phrase = \text{--}}			% for ranges: \numrange{1}{5} becomes 1--5 (long dash)


% LIST OF SYMBOLS (NOMENCLATURE)
\usepackage[notocbasic,						% I DON'T KNOW WHY, but it didn't work on Linux without this https://tex.stackexchange.com/questions/654662
english]
{nomencl}						% for the list of symbols
\makenomenclature
% to create sections automatically grouped in the List of Symbols (Nomenclature): https://www.overleaf.com/learn/latex/Nomenclatures#Grouping
\usepackage{etoolbox}
\renewcommand\nomgroup[1]{%
	\item[\bfseries
	\ifstrequal{#1}{A}{Physical quantities}{%
		\ifstrequal{#1}{B}{Superscripts}{%
			\ifstrequal{#1}{C}{Subscripts}{%
				\ifstrequal{#1}{D}{Other symbols}{}}}}%
	]}
% to include units in Nomenclature (https://tex.stackexchange.com/questions/452107/)
\newcommand{\nomunit}[1]{%
	\renewcommand{\nomentryend}{\hspace*{\fill}#1}%
}
\renewcommand{\nomname}{List of Symbols}	% instead of "Nomenclature", "List of Symbols" is used
\renewcommand{\nompreamble}{If not differently specified, thermodynamic properties are intended as static properties.}


% GLOSSARY
\usepackage[acronym,						
			%toc,							% add the list to the table of contents
			nomain,							% the Acronym page where the list is printed is not in the main THESIS.tex
			nonumberlist,					% without reference to the page of its first appearance
			nogroupskip,					% uniform vertical spacing of the entries
			nopostdot]						% without full stop at the end of the description of the acronym
			{glossaries}					% COMMAND: \acrshort{...} for the acronym, \acrlong{...} for the full name
\makeglossaries


% REFERENCES IN THE TEXT
\bibliographystyle{alpha}				% citation style
\usepackage{xurl}       				% url can be broken on several lines
%\usepackage{bigfoot}					% for CODES and URLS in footnotes (https://tex.stackexchange.com/questions/203)
\usepackage{hyperref}   				% clickable links
\hypersetup{
	colorlinks=true,    			% coloured links (otherwise they're in red boxes)
	citecolor=red,					% default is green, not that much visible
}
\usepackage[english]{cleveref}			% handy \cref{lab1,lab2} and \Cref{} commands
\usepackage{todonotes}
%\usepackage[nottoc,numbib]{tocbibind}


% FIGURES
\usepackage{graphicx}		% to include external images with \includegraphics{}
\usepackage{standalone}		% outsourcing figures


% ------------------------------------------------------------------------------
%							USER DEFINED COMMANDS
% ------------------------------------------------------------------------------


\newcommand{\lato}[1]{#1\marginpar{\footnotesize #1}}			% marginpar with small text
\newcommand{\nb}{\textbf{NB: }}								% bold "NB:"
\newcommand{\imp}[1]{\colorbox{yellow}{#1}}					% yellow highlight
% https://tex.stackexchange.com/questions/296121/
%\newcommand{\imp}[1]{\colorbox{yellow}{\parbox{\dimexpr\linewidth-2\fboxsep}{#1}}}
\newcommand{\prov}[1]{{\color{red} #1}}						% to be checked


\usepackage{mathsCommand}

% TEXT HIGHLIGHTING (to distinguish scope of informations in the text)
% (for the conditional compilation with \ifdefined: https://tex.stackexchange.com/questions/33576)
\newenvironment{theory}{\ifdefined\THEORY 		\color{Orange} 			\fi}{}	% theory (advanced)
\newcommand{\theo}[1]{\ifdefined\THEORY 		{\color{Orange} #1} 	\fi}
\newenvironment{advanced}{\ifdefined\ADVANCED 	\color{OliveGreen} 		\fi}{}	% advanced level/details
\newcommand{\adv}[1]{\ifdefined\ADVANCED 		{\color{OliveGreen} #1} \fi}
\newenvironment{basic}{\ifdefined\BASIC 		\color{RoyalBlue} 		\fi}{}	% basic information
\newcommand{\bas}[1]{\ifdefined\BASIC 			{\color{RoyalBlue} #1} 	\fi}


% CONDITIONAL COMPILATION: to control which "the level" of the notes, decomment only the desired level
% (https://tex.stackexchange.com/questions/33576)
\newcommand*{\BASIC}{}
\newcommand*{\THEORY}{}
\newcommand*{\ADVANCED}{}

\begin{document}
\title{CFD notes}
\author{Andrea Pinardi}
\date{\today \\[1ex] version \version}
\maketitle

Prova \cite{Qin2021} e prova \acrshort{tvd} and \acrfull{tvd}

% !TEX encoding = UTF-8 Unicode
% !TEX TS-program = pdflatex
% !TEX spellcheck = en-GB
% !BIB TS-program = bibtex
% !TEX root = ../MAIN.tex

\section{Maths}
\subsection{Numbering equations}
\[
\begin{sistema}
	x & = 1 + 2 \\
	y & = f(x) - 3
\end{sistema}
\]

Sistema con UN SOLO numero + graffa: \cref{eq.unica}
\begin{equation}\label{eq.unica}
\begin{sistema}
	x & = 1 + 2  \\
	y & = f(x) - 3
\end{sistema}
\end{equation}

Sistema con un numero DIVERSO per ogni eq numerate separatamente (ma graffa stretta): \cref{eq.Sist} e \cref{eq.Sista}
\begin{sistemaSub}
		x & = 1 + 2 \label{eq.Sist} \\
		y & = f(x) - 3 \label{eq.Sista}
\end{sistemaSub}

Numero COMUNE per ogni equazione + sottonumerazione (una label per l'intero sistema + una label per ogni eq.): la \cref{eq.energyBalance} contiene sia \cref{eq.EB1} che \cref{eq.EB2}
\begin{subequations}\label{eq.energyBalance}
	\begin{align}
		X 	& = Y + 2 \label{eq.EB1} \\ 
		l & = Z^2 - 3 \label{eq.EB2}
	\end{align}
\end{subequations}

{\color{red} DA PERFEZIONARE ANCORA: (vorrei fare la stessa cosa che in quello prima, ma con un ambiente -- ora non riesco a mettere la label unica esterna e nemmeno a mettere la graffa bella)
\begin{sistemaX}
			X 	& = Y + 2 \label{eq.EB3} \\ 
	l & = Z^2 - 3 \label{eq.EB4}
\end{sistemaX}
}

\subsection{Vectors and matrices}
\[
\vet{v} = \vet{\omega} \cross \vet{r}
\]
\[
\vet{v} = \vet{\omega} \scal \vet{r}
\]
\[
\norm{x} \qquad \norm*{\int x^2}
\]
\[
\abs{X} \qquad \abs*{\int x^2}
\]
\[
\mat{\Omega} = 
\begin{matrice}
	\vertbar & x & \vertbar \\
	y		& z & -x  \\
	\vertbar & x & \vertbar \\
\end{matrice}
\qquad
\mat{A} =
\begin{matrice}
    \horzbar & a^{T}_{1} & \horzbar \\[2ex]		% increase spacing
	\horzbar & a^{T}_{2} & \horzbar \\
			& \vdots    &          \\
	\horzbar & a^{T}_{n} & \horzbar
\end{matrice}
\]

\subsection{Derivatives}
\[
\du y
\]
Using the same command (default: 1st order)
\[
\der{f}{x} \qquad \der[2]{f}{x}
\]
Se lo si vuole fare inline c'è sia del 1st order: $\derline{y}{x}$ che di ordine superiore: $\derline[3]{y}{x}$
\[
y
\]
Idem, ma per quelle parziali:
\[
\parz{f}{x} \qquad \parz[2]{f}{x}
\]
e inline: $\parzline{y}{x}$ oppure $\parzline[4]{y}{x}$
\[
\grad f \qquad 
\dive f \qquad 
\curl \vet{v} \qquad
\lap v
\]
Per chiarezza si possono anche mettere le \{ \} attorno alla funzione (non cambia nulla per il comando)
\[
\grad{f} \qquad 
\dive{f} \qquad 
\curl{\vet{v}} \qquad
\lap{v}
\]
Derivata materiale:
\[
\Der{e}{t}
\]

\subsection{Miscellaneous}
Ora i comandi \verb|\epsilon| e \verb|\phi| producono quelle belle
\[
\epsilon \qquad \phi
\]
al posto degli aborti (che, non so perché, sono il default)
\[
\varepsilon \qquad \varphi
\]

Versione calligrafica delle lettere, maiuscole e minuscole (di default non ci sono le minuscole):
\[
\mathcal{G} \qquad \mathcal{g}
\]
\[
2 + 5 \ji \qquad \tilde{x} \qquad x = \cost \qquad \hat{x}
\]
Uguale per definizione:
\[
Y \defi \Delta H / \Delta L
\]
Media
\[
\avg{z}
\]
Gruppo adimensionale
\[
\Rey
\]
Underbrace con testo grande (leggibile)
\[
\sotto{\der{x}{y} - 3 + 12x - 7}{y^2}
\]
Big bracket for evaluation of derivatives:
\[
\eval{\der{y}{x}}_{2x}
\]

% ------------------------------------------------------------------------------
%								APPENDICES
% ------------------------------------------------------------------------------

\appendix
% !TEX encoding = UTF-8 
% !TEX TS-program = pdflatex
% !TEX spellcheck = en-GB
% !BIB TS-program = bibtex
% !TEX root = ../MAIN.tex

% PHYSICAL VARIABLES
\nomenclature[A]{$\Phi$}{Generic extensive property}
\nomenclature[A]{$\rho$}{Density \nomunit{\si{\kilogram\per\meter\cubed}}}
\nomenclature[A]{$u, v, w$}{Velocities \nomunit{\si{\meter\per\second}}}

% SUPERSCRIPTS
\nomenclature[B]{$'$}{Per unit length}

% SUBSCRIPTS
\nomenclature[C]{T}{Turbine}

% OTHER SYMBOLS
\nomenclature[D]{$\propto$}{Proportional to}



\printnomenclature
% !TEX encoding = UTF-8 
% !TEX TS-program = pdflatex
% !TEX spellcheck = en-GB
% !BIB TS-program = bibtex
% !TEX root = ../MAIN.tex


\newacronym{tvd}{TVD}{Total Variation Diminishing}

% force printing of all entries (if needed, otherwise just cite them in the text)
\glsaddall

\printglossary[type=\acronymtype]


% ------------------------------------------------------------------------------
%								BIBLIOGRAPHY
% ------------------------------------------------------------------------------


\bibliography{chapters/biblio}	% .bib file containing the bibliography


\end{document}

