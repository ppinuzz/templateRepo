% !TEX encoding = UTF-8 Unicode
% !TEX TS-program = pdflatex
% !TEX spellcheck = en-GB
% !BIB TS-program = bibtex

\documentclass{beamer}[10]
% se occorre creare gli HANDOUT (i.e. le slide organizzate in modo da poterle stampare senza avere 1 slide per foglio):
% \documentclass[handout]{beamer}									% aggiunta la keyword HANDOUT
% \usepackage{pgfpages}												% per stampare più slide per pagina
% \pgfpagesuselayout{2 on 1}[a4paper,border shrink=5mm]				% {FRAMES on 1 SHEET}
% \pgfpagesuselayout{4 on 1}[a4paper,border shrink=5mm,landscape]	% idem, ma ottimizzato per averne 4 per foglio
\usetheme{PaloAlto}
%\usecolortheme{crane}

% \usetheme			seleziona il tema che dà l'ASPETTO GENERALE della presentazione
% \usecolortheme 	seleziona il COLORE principale e il suo set di sfumature (e.g. ``crane'' è il giallo-arancio + le sue sfumature)
% \useinnertheme	style of the title and part pages, the itemize, enumerate, description, block, theorem and proof environments as well as figures, tables, footnotes and bibliography entries
% \useoutertheme	style of the head and footline, the logo, the sidebars and the frame title
% \usefonttheme		cambia il tema dei font

\usepackage[utf8]{inputenc}
\usepackage[T1]{fontenc}
\usepackage[italian]{babel}
\usepackage{microtype}

\usepackage{booktabs}
\usepackage{caption}

\begin{document}
	\title{My First Presentation}
	\subtitle{Using Beamer in 2023}
	\author{Andrea Pinardi}
	\institute{Politecnico di Milano}
	%\date{\today}
	
	% SLIDE = una singola pagina nel pdf finale
	% FRAME = insieme di ciò che viene creato in un singolo frame environment
	
	% usare \pause creare fisicamente più slide (i.e. più pagine nel pdf), che sono tutte caratterizzate dallo stesso numero in basso a destra nella presentazione (i.e. compongono lo stesso frame, che viene rivelato pezzo per pezzo)
	
	% OGNI slide è introdotta usando l'ambiente FRAME
	\begin{frame}
		\titlepage
	\end{frame}
	
	% indice iniziale
	\begin{frame}
		\frametitle{Outline}
		\tableofcontents
	\end{frame}
	
	\section{Tutorial 1}
	\subsection{sub a}
	\begin{frame}
		\frametitle{Titolo della slide}
		Lorem ipsum dolor sit amet, consectetur adipisicing elit, sed do eiusmod tempor incididunt ut labore et dolore magna aliqua.
		
		Tutto è tratto da \url{https://www.overleaf.com/learn/latex/Beamer_Presentations\%3A_A_Tutorial_for_Beginners_(Part_1)\%E2\%80\%94Getting_Started}
	\end{frame}
	
	\begin{frame}
		\label{label_della_slide}		% serve per il reindirizzamento con un hyperlink (cfr. tutorial 3)
		\frametitle{Pacchetti che beamer carica automaticamente}
		\begin{itemize}
			\item graphicx
			\item amsmath
		\end{itemize}
	\end{frame}
	
	% le liste funzionano come al solito
	\section{Tutorial 2}
	\begin{frame}
		\frametitle{Liste normali}
		\begin{enumerate}
			\item prova
			\item prova 2
			\begin{itemize}
				\item prova 3
				\item prova 4
			\end{itemize}
		\end{enumerate}
	\end{frame}
	
	\subsection{In numeri romani}
	\begin{frame}
		\frametitle{Liste romane}
		\begin{enumerate}[i]
			\item prova romana
			\item prova romana 2
			\begin{enumerate}[I]
				\item PROVA ROMANA
				\item PROVA ROMANA 2
			\end{enumerate}
		\end{enumerate}
	\end{frame}
	
	\subsection{Colonne}
	\begin{frame}
		\frametitle{Per ordinare in colonne}
		\begin{columns}
			\column{0.5\textwidth}	% da dare in % dello spazio per il testo
			Testo della prima colonna
			\column{0.2\textwidth}
			Testo della seconda colonna (non necessariamente larga come la prima)
		\end{columns}
	\end{frame}
	
	\begin{frame}
		\frametitle{Con una foto nella colonna}
		Se si vuole aggiungere un'immagine, NON serve caricare anche graphicx (lo carica già beamer)
		\begin{columns}
			\column{0.5\textwidth}
			Qui ci la descrizione della foto a destra
			\column{0.5\textwidth}
			\includegraphics[scale=0.1]{prova}
		\end{columns}
	\end{frame}
	
	\begin{frame}
		\frametitle{Con una foto al centro}
		Usando invece il l'ambiente figure si mette anche la caption:
		\begin{figure}
			\includegraphics[scale=0.2]{prova}
			\caption{Adoro questa canzone}
		\end{figure}
	\end{frame}
	
	\subsection{Tabelle}
	\begin{frame}
		\frametitle{Tablle normali}
		Funzionano come sempre: la tabella vera e propria la sia crea con TABULAR, poi con TABLE la si rende flottante e si possono aggiungere anche altri opzioni (e.g. la caption)
		\begin{table}
			\begin{tabular}{lcc}
				\toprule
				Nome & Anno & Culo \\
				\midrule
				Sei & 7 & sehila \\
				Sei & 7 & sehila \\
				Sei & 7 & sehila \\
				Sei & 7 & sehila \\
				\bottomrule
			\end{tabular}
			\caption{Caption della tabella}
		\end{table}
	\end{frame}
	
	\section{Tutorial 3}
	
	\subsection{Blocks}
	\begin{frame}
		\frametitle{Base}
		Per le aree di testo riquadrate, usare l'ambiente BLOCK:
		\begin{block}{Titolo del block}
			Come ai tempi di Statistica
		\end{block}
		Per avere la versione in rosso, usare ALERTBLOCK:
		\begin{alertblock}{Titolo del block rosso}
			Warning vari qui
		\end{alertblock}
	\end{frame}
	
	\begin{frame}
		\frametitle{Per la matematica}
		Per avere invece delle definizioni, usare DEFINITION come block:
		\begin{definition}
			Def. di qualcosa
		\end{definition}
		e, analogamente, per avere degli esempi usare EXAMPLE:
		\begin{example}
			Esempio di esercizio
		\end{example}
	\end{frame}
	
	\begin{frame}
		\frametitle{Per la matematica - 2}
		Analoghi, sono gli ambienti ``già nominati'' THEOREM:
		\begin{theorem}[Nome del teorema]
			Sia $\sum_{i=1}^N f(x_i)$ una successione...
		\end{theorem}
		l'ambiente COROLLARY:
		\begin{corollary}[Del teo che voglio]
			Culo
		\end{corollary}
		e l'ambiente PROOF:
		\begin{proof}
			Dimostrazione...
		\end{proof}
		Sia THEOREM che COROLLARY, di default, sono in corsivo
	\end{frame}
	
	\subsection{Codice}
	\begin{frame}[fragile]		% serve per inserire il codice
		\frametitle{Snippet di codice}
		Per il codice occorre:
		\begin{enumerate}
			\item usare l'argomento opzionale FRAGILE per l'ambiente frame della slide
			\item usare l'ambiente SEMIVERBATIM
			\item per alcuni caratteri occorre inserire un backslash prima:
			\begin{itemize}
				\item backslash
				\item parentesi graffe
			\end{itemize}
		\end{enumerate}
		Snippet in C:
		\begin{semiverbatim}
			#include<stdio.h>
			int main()\{
			
			return 0;
			\}
		\end{semiverbatim}
	\end{frame}
	
	\subsection{Hyperlinks e buttons}
	\begin{frame}
		\frametitle{Mettere hyperlink}
		Nelle slide (frames) che si vuole collegare, occorre aggiungere una LABEL subito dopo il frame (cfr. la slide con l'elenco dei pacchetti, che puoi vedere cliccando \hyperlink{label_della_slide}{PROPRIO QUI})
		
		Se vuoi essere fancy, puoi usare un buttons anziché l'hyperlink normale, usando un BEAMERBUTTON, come in questo comando: \hyperlink{label_della_slide}{\beamerbutton{BOTTONE SLIDE}}
		
		Altri tipi di bottoni possibili sono:
		\begin{itemize}
			\item il GO TO: \hyperlink{label_della_slide}{\beamergotobutton{BOTTONE GO-TO}}
			\item lo SKIP: \hyperlink{label_della_slide}{\beamerskipbutton{BOTTONE GO-TO}}
			\item il RETURN: \hyperlink{label_della_slide}{\beamerreturnbutton{BOTTONE GO-TO}}
		\end{itemize}
		che ti riportano tutti nello stesso posto (il nome diverso è solamente ``estetico'', non c'è differenza pratica come effetto)
	\end{frame}
	
	\section{Pausa}
	\begin{frame}
		\frametitle{Liste}
		\begin{itemize}
			\pause
			\item Point A
			\pause
			\item Point B
			\begin{itemize}
				\pause
				\item part 1
				\pause
				\item part 2
			\end{itemize}
			\pause
			\item Point C
			\pause
			\item Point D
		\end{itemize}
	\end{frame}
	
	% OVERLAY SPECIFICATION = vengono date con le parentesi uncinate <>, indicando tra le parentesi su quale *slide* (NON frame) dovrebbe apparire l'elemento
	% <3->			appare dalla slide 3 in poi
	% <-3>			appare dalla prima slide alla 3, poi non più
	% <-5,7-9,12>	appare dalla slide 1 alla 5, poi appare solamente sulle slida dalla 7 alla 9 e sulla slide 12
	
	\section{Overlay specifications}
	\begin{frame}
		\frametitle{Overlay spec.}
		\begin{enumerate}
			\item<1-> Point A
			\item<2-> Point B
		\end{enumerate}
	\end{frame}
	
	\begin{frame}
		\frametitle{Overlays trasparenti}
		\onslide<1->{First Line of Text}
		
		\setbeamercovered{transparent}		% TUTTO il testo dell'overlay indicato da \onslide, da qui in poi, apparirà come trasparente
		\onslide<2->{Second Line of Text}
		
		\onslide<3->{Third Line of Text}
	\end{frame}
	
	\begin{frame}
		\frametitle{Overlays trasparenti 2}
		\onslide<1->{First Line of Text}
		
		\setbeamercovered{transparent}
		\onslide<2->{Second Line of Text}
		
		\setbeamercovered{invisible}		% se voglio che questo invece sia invisibile, devo esplicitamente reimpostarlo
		\onslide<3->{Third Line of Text}
	\end{frame}
	
	% Altri comandi simili:
	%	\visible 	stesso effetto di \uncover, però lascia uno spazio bianco sulla slide in QUALSIASI caso (anche se si usa \setbeamercovered{transparent})
	%	\invisible	opposto di \visible
	%	\only 		stesso effetto di \visibile, però NON occupa spazio (i.e. il testo appare sempre ``nella stessa posizione'', come se ogni volta la frase venisse sostituita da un'altra)
	
	\begin{frame}
		\frametitle{Overlays trasparenti 3}
		\only<1>{First Line of Text}
		
		\only<2>{Second Line of Text}
		
		\only<3>{Third Line of Text}
	\end{frame}
	
	% i comandi per il text formatting pensati per funzionare con gli overlay restituiranno:
	%	- testo normale quando agiranno su slide non dichiarate nelle specifications (i.e. tra le <> ?)
	%	testo formattato come indicato, negli altri casi
	
	% Le overlay specifications SPESSO funzionano anche con gli ambienti, però la specification tra le <...> va messa DOPO il nome dell'ambiente tra le graffe
	
	\begin{frame}
		\frametitle{Maths Blocks}
		\begin{theorem}<1->[Pythagoras] 
			$ a^2 + b^2 = c^2$
		\end{theorem}
		\begin{corollary}<3->
			$ x + y = y + x  $
		\end{corollary}
		\begin{proof}<2->
			$\omega +\phi = \epsilon $
		\end{proof}
	\end{frame}
	
	\section{Tabelle riga-per-riga animate}
	\setbeamercovered{invisible}
	\begin{frame}
		\frametitle{Tables}
		\begin{table}
			\begin{tabular}{l | c | c | c | c }
				Competitor Name & Swim  & Cycle & Run   & Total              \\ \hline\hline
				John T          & 13:04 & 24:15 & 18:34 & 55:53 \onslide<2-> \\
				Norman P        & 8:00  & 22:45 & 23:02 & 53:47 \onslide<3-> \\
				Alex K          & 14:00 & 28:00 & n/a   & n/a \onslide<4->   \\
				Sarah H         & 9:22  & 21:10 & 24:03 & 54:35
			\end{tabular}
			\caption{Triathlon results}
		\end{table}
	\end{frame}
	
	\section{Tutorial 4}
	
\end{document}

%\usepackage{pgf}
%\usepackage{beamerthemesplit}

%\definecolor{kugreen}{RGB}{50,93,61}
%\definecolor{kugreenlys}{RGB}{132,158,139}
%\definecolor{kugreenlyslys}{RGB}{173,190,177}
%\definecolor{kugreenlyslyslys}{RGB}{214,223,216}
%\setbeamercovered{transparent}
%\mode<presentation>
%\usetheme[numbers,totalnumber,compress,sidebarshades]{PaloAlto}
%\setbeamertemplate{footline}[frame number]
%
%\usecolortheme[named=kugreen]{structure}
%\useinnertheme{circles}
%\usefonttheme[onlymath]{serif}
%\setbeamercovered{transparent}
%\setbeamertemplate{blocks}[rounded][shadow=true]
%
%%\logo{\includegraphics[width=1.5cm]{billeder/logo}}
%\title{Dæmpning af seismiske bølger}
%\author{Thomas R. N. Jansson}
%\institute{Niels Bohr Institute \\ University of Copenhagen}
%\date{21 December 2007}